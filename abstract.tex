\begin{abstract}

  Instead of the current trend consisting of building larger and
  larger data centers (DCs) in few strategic locations, the DISCOVERY
  initiative\footnote{\url{http://beyondtheclouds.github.io}}
 proposes to leverage any network point of presences (PoP, \ie a
  small or medium-sized network center) available through the
  Internet. The key idea is to demonstrate a widely distributed Cloud
  platform that can better match the geographical dispersal of users.
  This involves radical changes in the way resources are managed, but
  leveraging computing resources around the end-users will enable to
  deliver a new generation of highly efficient and sustainable Utility
  Computing (UC)
  platforms, thus providing a strong alternative to the actual Cloud
  model based on mega DCs (i.e. DCs composed of tens of thousands
  resources).

  Critical to the emergence of such distributed Cloud platforms is the
  availability of appropriate operating mechanisms. Although, some of
  protagonists of Cloud federations would argue that it might be
  possible to federate a significant number of micro-Clouds hosted on
  each PoP, we emphasize that federated approaches aim at delivering a
  brokering service in charge of interacting with several Cloud
  management systems, each of them being already deployed and operated
  independently by at least one administrator. In other words, current
  federated approaches do not target to operate, remotely, a
  significant amount of UC resources geographically distributed but
  only to use them. The main objective of DISCOVERY is to design,
  implement, demonstrate and promote a unified system in charge of
  turning a complex, extremely large-scale and widely distributed
  infrastructure into a collection of abstracted computing resources
  which is efficient, reliable, secure and friendly to operate and
  use.

  After presenting the DISCOVERY vision, we explain the different
  choices we made, in particular the choice of revising the OpenStack
  solution leveraging P2P mechanisms. We believe that such a strategy
  is promising considering the architecture complexity of such systems
  and the velocity of open-source initiatives.

%
%   By deploying the concept of micro/nano data-centers directly in network point
%   of presences, it becomes possible to deliver a new generation of
%   cloud computing infrastructure more efficient and sustainable. Among
%   the different challenges we need to address to favor the adoption of such a model, the design and the
%   implementation of a system in charge of turning such a complex and diverse
%   network of resources into a collection of abstracted computing
%   facilities that are convenient to administrate and use is critical.
% % Although a reference architecture describing fundamental services constituting
% % an IaaS manager has been proposed, a detailed overview
% % of the mechanisms that are needed to build a massively distributed cloud is
% % still missing.
% %
%   In this paper, we introduce the premises of such a system. The main
%   contribution of our work is that we did not implement our
%   prototype from scratch but instead slightly revisited the OpenStack
%   solution. We believe that such a strategy is promising
%   considering the architecture complexity of such systems and the velocity of open-source
%   initiatives.
%   We present few preliminary scenarios that enabled us to validate the
%   correct behavior of our prototype across distinct servers/sites of
%   the Grid'5000 testbed.
\keywords{Cloud computing, IaaS Architecture, OpenStack, NoSQL, Peer
  to Peer}
\end{abstract}
